\section{Partitionnement \'etendu: Formater les partitions}
\includegraphics[scale=0.45]{/mdk/doc/manual/screenshots/fr/fdisk-extended.png} \\
Choisissez la partition que vous voudriez formater en s\'electionnant le type du syst\`eme de fichier et cliquant sur $\ll$Formater$\gg$. Quant au choix du syst\`eme de fichier $\ll$correct$\gg$, regardez l'information en bas de cette page.\\
Quand vous avez format\'e les partitions souhait\'es, cliquez sur $\ll$Pas suivant (choisir le but de l'installation)$\gg$.\\
\subsection{Information suppl\'ementaire}
Il faut au moins une partition format\'e ext2, ext3 ou reiserfs et une partition avec un syst\`eme de fichier linux-swap pour pouvoir commencer avec l'installation.\\
\subsection{Information: Syst\`emes de fichier}
\begin{enumerate}
\item ext2: un syst\`eme de fichier plus \^ag\'e, apte \`a l'installation du syst\`eme de base\\
\item ext3(recommand\'e), reiserfs: syst\`emes de fichier journalling plus modernes, qui sont aptes \`a l'installation du syst\`eme de base \\
\item linux-swap: est utilis\'e pour l'\'etablissement de la m\'emoire de permutation linux\\
\end{enumerate}

\chapter{Voraussetzungen f�r m23}
Zur Integration des m23-Software-Verteilungssystems in Ihr Netzwerk ben�tigen Sie folgendes:
\begin{itemize}
\item m23-Server
\item m23-Client(s)
\item Internetzugang
\item PC mit Webbrowser
\item m23-Server-Installations-CD
\end{itemize}

\section{Der m23-Server}
Der m23-Server �bernimmt die komplette Verteilung der Software und erf�llt verschiedene Aufgaben:
\begin{itemize}
\item Generierung der Administrationsoberfl�che
\item Bereitstellung von Installations-Skripten zur Laufzeit
\item Softwarepakete-Cache
\item Datenbank
\item Boot-Server
\item DHCP-Server
\end{itemize}
Der m23-Server sollte je nach Anzahl der Clients mit ausreichend Arbeitsspeicher und einem angemessenen Prozessor ausgestattet werden. F�r wenige Clients oder zum Ausprobieren reicht ein PC �lterer Bauart (z.B. PI mit 166MHz und 64MB RAM). F�r gr��ere Netzwerke sollten Sie einen neueren PC mit mindestens 256MB RAM einsetzen. In beiden F�llen ben�tigt Ihr Server eine Netzwerkkarte und eine Festplatte ab 5GB. Bei der Wahl der Festplatte sollten Sie darauf Wert legen, den zur Verf�gung stehenden Speicherplatz nicht zu knapp zu bemessen. Bei Einsatz mit vielen Clients sollten Sie auf ausreichenden Datendurchsatz der Festplatte achten. Da der m23-Server die Festplatte zum Cachen der Softwarepakete benutzt, kann es bei der Verteilung von vielen verschiedenen Softwarepaketen und zu klein bemessener Festplatte zu Fehlern kommen.
\begin{itemize}
\item CPU: ab 166Mhz (empfohlen 1GHz)
\item RAM: ab 64MB (empfohlen ab 256MB)
\item Festplatte: ab 5GB
\item Netzwerkkarte: empfohlen 100MBit/1GBit
\end{itemize}

\section{m23-Clients}
F�r die Clients gilt im Prinzip das gleiche. Um die Softwareverteilung zu vereinfachen, sollten Sie darauf achten, da� die Clients Wake-On-Lan-f�hig sind, da m23 die Clients dann eigenst�ndig f�r Installationsauftr�ge starten und nach Abschlu� wieder in den Ruhezustand versetzen kann. Auch ist der Einsatz von Netzwerkkarten, die ein Bootrom enthalten, empfehlenswert, da die Clients so �ber das Netzwerk gebootet werden k�nnen, ohne da� sie per Bootdiskette/CD gestartet werden m�ssen. Dies ist allerdings nur f�r das Aufspielen des Betriebbsystems erforderlich. m23 unterst�tzt PXE und Etherboot als Bootstandards.

\section{Internetzugang}
F�r m23 ben�tigen Sie einen Internetzugang, da die ganzen Softwarepakete nicht auf der CD enthalten sind, sondern aus dem Internet heruntergeladen und st�ndig aktualisiert werden.

\section{PC mit Webbrowser}
F�r die Administration ben�tigen Sie einen Webbrowser, mit dem Sie auf den Server zugreifen k�nnen.

\section{Die m23-Server-Installations-CD}
Um einen m23-Server aufzusetzen, ben�tigen Sie die m23-Server-Installations-CD die es zum kostenlosen Download auf \underline{http://m23.sf.net} gibt. Brennen Sie die ISO-Datei einfach mit einem Brennprogramm und booten Sie danach den Server davon.
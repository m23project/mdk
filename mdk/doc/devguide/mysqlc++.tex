\section{how to access mysql databases with C++?}
\subsection{Introduction}
In this little howto you wil learn how to set up KDevelop to write a simple application that is able to connect to a mysql server and execute sql-statements on it. I will only discuss how to write to and not to read from a database. If you want to read something, look at the mysql C-API that can be found at: \textbf{http://www.mysql.com/doc/en/C.html}

\subsection{What do you need?}
You should have installed the following applications and libraries:

\begin{itemize}
\item kdevelop
\item libmysqlclient10-dev, libmysqlclient10
\item libc6-dev
\end{itemize} 

\subsection{Preparing the server}
\subsubsection{Enable network access}
To get access to the server over the network you need to do a little adjustment:
Open \textit{/etc/mysql/my.cnf} and comment out the line \texttt{skip-networking} so that it looks like \texttt{\#skip-networking}.
Don't forget to save ;) Now you have to restart the mysql server.\\\\
I do it this way:\\
\texttt{
killall -9 safe\_mysqld\\
killall -9 mysqld\\
/etc/init.d/mysql start\\
}

\subsubsection{Adding a database user}
You have enabled network access, now you need a database user. The new user should be able to do some changes to tables (insert, delete and update). For creating the user I use phpMyAdmin, a very powerful tool for all ways of modifying mysql databases. If you don't allready know that tool, have a look at: \textbf{http://www.phpmyadmin.net/}.
\begin{itemize}
\item log in to your m23 server: open \textbf{http://<serverIP>/m23admin/phpMyAdmin/index.php} in a webbrowser and adjust <serverIP>  to the ip address of your m23 server.
\item select "home" on the left
\item click on "Privileges" in the main window
\item click "Add a new User"
\item think about a good user name and insert it at: "User name:"
\item for "Host:" you can leave it on "any host" and select "\%" for the host name. If you want only selected hosts to log in, you can enter the ip or hostname.
\item select a password and retype it in the text field below
\item in "Global privileges" -> "Data" select at least: INSERT, UPDATE and DELETE
\item click the "GO" button to accept your changes
\item select "home" on the left
\item click on "Reload MySQL" in the main window
\end{itemize}
After this adjustments you should be able to log in to the m23 server with your new account.

\subsection{Preparing KDevelop}
\subsubsection{Creating a project}
In the KDevelop GUI select "Project" -> "New" select from the list "C++" under "Terminal". You have to select "C++", "C" will not work!
\begin{itemize}
\item select "Next"
\item select what you want in the next dialog pages
\end{itemize}

\subsubsection{Adjusting project properties}
Select "Project" -> "Options" -> "Compiler options" -> "Linker (tab)".
Now add \textbf{-lmysqlclient -lcrypt -lnsl} to the input line. Close the window with a click on the "OK" button. These settings are required to build applications with mysql support.

\subsection{Filling main.cpp}
After all preparations we step forward to the source code. You need to include \textbf{include <mysql/mysql.h>} and create a global variable called mysql from the type MYSQL.

\subsubsection{Connecting to the server}
\begin{verbatim}
#include <mysql/mysql.h>

MYSQL connection;

void connect(char *server, char *dbuser, char *passw)
{
  if (!mysql_connect(&connection,server,dbuser,passw))
  {
      fprintf(stderr, "Failed to connect to database:\
              Error: %s\n",mysql_error(&connection));
      exit(2);
  }
  mysql_select_db(&connection,"m23");
};
\end{verbatim}
\begin{enumerate}
\item First of all we have to include the necessary header file: \textit{\#include <mysql/mysql.h>}
\item The mysql\_connect function requires a \textbf{MYSQL} variable (here called
\textit{connection}) to store the connection data, a server name or ip, a valid mysql user and proper password. If all works correctly, mysql\_connect will return 0, otherwise we get an error, print an error message and quit.

\item mysql\_error will generate a human readable error message.

\item At least we should choose our database, this is done with mysql\_select\_db. Now the connection should be established.

\end{enumerate}





\subsubsection{Disconnecting from the server}
\begin{verbatim}
void disconnect()
{
 mysql_close(&connection);
};
\end{verbatim}
This function disconnects from the server. The global variable \textit{connection} is used to identify the right connection to the server.





\subsubsection{Send SQL statements}
\begin{verbatim}
MYSQL_RES *query(char *sql)
{
 MYSQL_RES *res;

 if (mysql_query(&connection, sql) != 0)
    {
     fprintf(stderr, "Failed to send query: Error: %s\nsqlstatement:%s\n",\
     mysql_error(&connection),sql);
     exit(2);
    };

 res = mysql_store_result(&connection);

 return(res);
};
\end{verbatim}
\begin{enumerate}
\item We declare a variable (\textit{res})that will held the result of our query
\item mysql\_query requires a established connection to a mysql server and a valid sql statement. If the statement gets executed corretly, it returns 0, otherwise you can use the mysql\_error function to get the error message.
\item And at last mysql\_store\_result stores the result of our query or \textbf{NULL} if there was an error.
\end{enumerate}





\subsection{Example}
\begin{verbatim}
int main(int argc, char *argv[])
{
  MYSQL_RES *res;
  unsigned int fields;
  my_ulonglong rows;
  MYSQL_ROW row;

  connect("127.0.0.1", "root", "");

  res=query("SELECT * FROM `clients`");

  fields=mysql_field_count(&connection); //<<1

  rows=mysql_num_rows(res); //<<2


  for (int i=0; i < rows; i++)
    {
        unsigned long *lengths;

        row=mysql_fetch_row(res); //<<3

        lengths = mysql_fetch_lengths(res); //<<4

        for (int fnr=0; fnr < fields; fnr++)
            printf("[%i: %s] ", (int) lengths[fnr],
                    row[fnr] ? row[fnr] : "NULL"); //<<5
    }

  mysql_free_result(res); //<<6

  disconnect();

  return EXIT_SUCCESS;
}
\end{verbatim}
\begin{enumerate}
\item Get the number of fields of the table.
\item Get the amount of result rows.
\item Fetch a row (one by one) and store the result in \textit{row}.
\item Store the lengths (in characters) of the result fields.
\item Print the fields of one row.
\item Free the memory by deleting the result data.
\end{enumerate}


\subsection{The complete program}
\begin{verbatim}
#include <mysql/mysql.h>
#include <stdio.h>
#include <stdlib.h>

MYSQL connection;

void connect(char *server, char *dbuser, char *passw)
{
  if (!mysql_connect(&connection,server,dbuser,passw))
  {
      fprintf(stderr, "Failed to connect to database:\
              Error: %s\n",mysql_error(&connection));
      exit(2);
  }
  mysql_select_db(&connection,"m23");
};





void disconnect()
{
 mysql_close(&connection);
};




     
MYSQL_RES *query(char *sql)
{
 MYSQL_RES *res;

 if (mysql_query(&connection, sql) != 0)
    {
     fprintf(stderr, "Failed to send query: Error: %s\nsqlstatement:%s\n",\
     mysql_error(&connection),sql);
     exit(2);
    };

 res = mysql_store_result(&connection);

 return(res);
};





int main(int argc, char *argv[])
{
  MYSQL_RES *res;
  unsigned int fields;
  my_ulonglong rows;
  MYSQL_ROW row;

  connect("127.0.0.1", "root", "");

  res=query("SELECT * FROM `clients`");

  fields=mysql_field_count(&connection); //<<1

  rows=mysql_num_rows(res); //<<2


  for (int i=0; i < rows; i++)
    {
        unsigned long *lengths;

        row=mysql_fetch_row(res); //<<3

        lengths = mysql_fetch_lengths(res); //<<4

        for (int fnr=0; fnr < fields; fnr++)
            printf("[%i: %s] ", (int) lengths[fnr],
                    row[fnr] ? row[fnr] : "NULL"); //<<5
    }

  mysql_free_result(res); //<<6

  disconnect();

  return EXIT_SUCCESS;
}
\end{verbatim}
\section{How to make network bootimages?}
\subsection{Introduction}
What are bootimages? Bootimages are used to boot up a client over the network. The bootimage contains a Linux kernel, an initial ramdisk and a script for fetching the jobs for the client.\\
A client runs for first start up thru the following sequence:\\
\begin{itemize}
\item get IP, netmask, bootimage name, ... from the m23 DHCP server
\item fetch the bootimage from the server with TFTP protocoll
\item load and extract the image to the client memory
\item start the contained kernel
\item kernel loades the included initrd
\item start up script fetches IP, netmask, etc. from the server again
\item the script fetches the first job for the client and saves the job script file
\item the script is executed and the next job will be fetched from the server
\end{itemize} 
This technology makes it possible to install angelOne Linux on an empty computer. No installed operating system is needed for partition, format or installation.
\subsection{creating a new bootimage}
In the /mdk/bootimage directory you can find the mkBootImage.sh script that generates the bootimage for PXE and Etherboot standard. The bootimages will be stored in /m23/tftp/ as m23pxeinstall (PXE kernel), initrd.gz (initial ramdisk for PXE) and m23install (Etherboot). To generate the Etherboot files you need to install \textbf{mknbi}.\\
\subsubsection{How it works?}
it creates a blank file of 17MB size
\begin{verbatim}
> dd if=/dev/zero of=initrd bs=1k count=23000
\end{verbatim}
sets up a loop device
\begin{verbatim}
> losetup /dev/loop1 initrd
\end{verbatim}
the 23MB file will be formated with EXT2
\begin{verbatim}
> mke2fs -m 0 -N 10000 /dev/loop1
\end{verbatim}
mount the file
\begin{verbatim}
> mount /dev/loop1 mnt
\end{verbatim}
copy needed files from root2 and hardware informations to the mounted image
\begin{verbatim}
> cp -rdpR root2/* mnt/
> cp -rdpR /usr/share/hwdata/*  root2/usr/share/hwdata/
\end{verbatim}
umnount the image
\begin{verbatim}
> umount mnt
\end{verbatim}
set "down" the loop device
\begin{verbatim}
> losetup -d /dev/loop1
\end{verbatim}
set correct file permissions
\begin{verbatim}
> chmod 0555 bzImage
> chown root:root bzImage
\end{verbatim}
set boot device in kernel
\begin{verbatim}
> rdev bzImage /dev/ram0
\end{verbatim}
generate bootimage for Etherboot
\begin{verbatim}
> mknbi-linux bzImage --first32pm --output=/m23/tftp/m23install --ip=dhcp
	--rootdir=/dev/ram0 initrd
\end{verbatim}
generate files for PXE
\begin{verbatim}
> cp bzImage /m23/tftp/m23pxeinstall
> gzip initrd
> mv initrd.gz /m23/tftp/initrd.gz
\end{verbatim}

\subsection{What can you do with this SDK?}
Modify all files in root2. These files are the files for a kind of Mini Linux distribution. You should be familar to Linux, if you want to change a thing. linurc will be the first script executed after network boot. If you want to do automatic execution this is the right place. \subsubsection{Build a new kernel}
Copy your new kernel to bzImage in the SDK directory and don't forget to copy the modules to root2/lib/modules. In the m23client-Install*.conf file you get the configuration for the kernel we used.
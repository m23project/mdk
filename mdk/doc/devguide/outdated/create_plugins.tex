\section{How to develop plugins for m23?}
\subsection{Introduction}
m23 gives you a plugin interface, that enables you to create your own plugins easily. A plugin can be everything that can be written in PHP and bash script. The plugins are packaged in an archive, that contains installation and deinstallation routines and of course the data for the plugin.

\subsubsection{What is this document about?}
You will learn what m23 plugins are and how to build them.

\subsection{How to create a plugin?}
First of all develop you plugin application and don't forget to test it ;) When all seems to be correct, create a new directory for your plugin package. This will be called \textbf{project directory}. 

\subsubsection{XXX.m23plg file}
This is your main plugin file, it should have a good chosen name and nothing like plugin.m23plg.\\\\
It may look like this:\\
\texttt{<?PHP\\
/*\\
pluginName:Testplugin2\\
pluginPage:testpage\\
pluginAuthor: me\\
pluginVersion: 1\\
pluginUpdate: www.pc-kiel.de/plugin.dat\\
pluginClientRequires: rdiff\\
pluginIcon: /gfx/overview-mini.png\\
*/\\
Here stands your code\\
?>}\\\\
The lines between /* and */ are treated by PHP as comments, but the m23admin uses this informations for:
\begin{itemize}
\item pluginName: is the name that is shown in the m23 menu
\item pluginPage: this name is used for internal purpose, this is how the page is known to m23
\item pluginAuthor: here you can leave your name, contact address etc.
\item pluginVersion: version number of your plugin
\item pluginUpdate: specifies the location where to find the file, containing informations about newer versions of the plugin
\item pluginClientRequires: if your plugin needs some packages installed on the client, you have to list these packagenames here
\item pluginIcon: you can select an icon to be shown before the menu entry
\end{itemize} 

\subsection{Installation}
If you install a plugin with the m23-interface the plugin package will be downloaded to /m23 and extracted. Before the real installation it will be checked that no files get overwritten. The user can select, if he wants to overwrite files or skip installation. Your plugin should have at least one XXX.m23plg file, where XXX stands for your plugin filename. You should organize your plugin package in a way, that this file will be extracted to one of these directories:
\begin{itemize}
\item /m23/data+scripts/m23admin/clients
\item /m23/data+scripts/m23admin/groups
\item /m23/data+scripts/m23admin/packages
\item /m23/data+scripts/m23admin/server
\item /m23/data+scripts/m23admin/tools
\end{itemize}
In other words:\\
your plugin package should have a directory structure like:\\ 

\texttt{data+scripts/m23admin/clients/myPlugin.m23plg}.\\\\
Copy all files of your plugin application with correct pathes to the \textbf{project directory}. For an easier way to make a m23 package have a look at the hint section.\\\\
Your project directory may look like this:\\
\texttt{/projectDir/data+scripts/m23admin/clients/myPlugin.m23plg\\
/projectDir/bin/myExecutable\\
/projectDir/data+scripts/myPluginDir/some data files\\
/projectDir/data+scripts/myPluginDir/some more scripts.php}

\subsubsection{install.sh}
If you want to do some additional installation e.g. install an extra \textit{deb package} on the server machine, this is the right place. In this file you can do whatever you want, but you shouldn't do something that will eventually crash the server.\\
During installation your plugin package will be extracted and afterwards install.sh will be executed. When execution is done, install.sh will be deleted.\\\\
Your install.sh may look like:\\
\texttt{
\#!/bin/bash \\
apt-get update\\
apt-get install desiredPackage}

\subsection{Deinstallation}
All your plugin files will be removed and you have the possibillity to do some additional clean up.

\subsubsection{deinstall.sh}
This is the place to add your deinstallation script. It is similar to the install.sh, you can remove packages and do other things, but be again careful not to crash the server.\\\\
Your deinstall.sh may look like:\\
\texttt{
\#!/bin/bash \\
apt-get update\\
apt-get remove desiredPackage}

\subsection{Update information files}
With the update information files you can specify name, version, size, description and the url where the update can be found. These informations are needed, if you want to update an existing plugin with the m23 admin. Name, version and description are free form. Size should be the extracted size of the plugin and url is the address, where the plugin data can be downloaded.
\subsubsection{How does an information file look like?}
\begin{verbatim}
pluginURL: http://m23.sf.net/testplug.tb2
pluginVersion: 2
pluginSize: 1023
pluginName: Testplugin
pluginDescription: This is only a test plugin
\end{verbatim} 

\subsubsection{Protocols for downloading files}
You can use http, ftp and local files for downloading your plugin information.

\subsection{Hints}
\subsubsection{Better naming for your files}
You normally will develop your application in the /m23 directory. To make finding your files easier you should name your files like \textit{BACKUP\_startpage.php}. Your files should always begin with your selected prefix.
Now you can find your files with:
\begin{verbatim}
find /m23/ -name BACKUP*
\end{verbatim} 
To make a package with the found files, simply use the following script:
\begin{verbatim}
for file in `find . -name BACKUP_*`
do
 files=$files" "$file
done
tar cvj -f myPackage.tb2 $files
\end{verbatim}
Where you replace \textit{BACKUP\_} with your package prefix and \textit{myPackage} with your package name.

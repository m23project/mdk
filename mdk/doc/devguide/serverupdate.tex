\section{How to create the server update files?}
m23 has a mechnism that makes it easy for the user to update the m23 server. The information for the update is fetched from the internet. There is a php script that generates the needed update information from single files. This files contain information about the new codename, the new version number, a describing text, a script that is exectued at the beginning and at the end of the update.
\subsection{The files}
All files for an update have to begin with the version number (e.g. 0.4.5):
\begin{itemize}
\item 0.4.5.begin: conatins the bas script that is executed a the beginning
\item 0.4.5.end: conatins the bas script that is executed a the end
\item 0.4.5.info: contains only the codename of the m23 release (e.g. shiver)
\item 0.4.5.text: helds the decribing text of the update. There you should put a changelog
\end{itemize} 
\subsubsection{xxx.begin}
This is only an example, don't assume it will work.
\begin{verbatim}
wget http://m23.sf.net/newdata.tb2
tar xfvj newdata.tb2
\end{verbatim}
\subsubsection{xxx.end}
\begin{verbatim}
rm newdata.tb2
\end{verbatim}

\subsubsection{xxx.info}
\begin{verbatim}
shiver
\end{verbatim}

\subsubsection{xxx.text}
You can use html tags in your release information.
\begin{verbatim}
This is the new shiver update. There will be the following changes:
<ul>
 <li> bigger</li>
 <li> louder</li>
 <li> ...
</ul>
\end{verbatim} 
\section{Introduction}
The MDK has got a menue system to make it easier for you to change things and to create your own m23 version, boot CDs etc.. In earlier versions of the MDK there were some widespread scripts without an user interface. The new menues make it easier to find what you are searching for. The menue system should be self-explanatory now.\\\\

Some of the things you can do with the MDK menues:
\begin{itemize}
\item Create server installation CDs: Build the operating system data file, the m23 programm file, the MDK file and make an installable boot CD from it.
\item Create bootimages for the clients: Build netboot images for PXE and Etherboot standard and create bootable client installation CDs.
\item Build special Debian packages: These packages are needed to add missing functionalities to the clients.
\item Build documentation: Tools for generating the "m23 manual" in different languages and the "Development guide". This includes generation of screenshots, PDF and HTML files. The text is fetched from the online help, source codes and other sources.
\end{itemize}

\section{Starting the MDK menue system}
All you have to do is to run the following command from a console:
\begin{verbatim}
/mdk/bin/menuStart
\end{verbatim}
\underline{Hint}: If this doesn't work install the \textbf{dialog} package, that is needed to draw the menues.
\section{Using new Debian/Ubuntu releases with m23}
To fully support a new Debian/Ubuntu release a few steps are needed. The following guide shows a generic approach to not miss an important step. The total time and difficulty of a new release depends on the changes the distribution made between the last release supported by m23 and new release and if there are new desktops etc. that should be supported.

\subsection{Test and development}
\begin{itemize}
\item Build a compressed root file system for the new release via /mdk/m23helper/compressedDebootstrap and put it (for testing) into the directory /m23/data+scripts/packages/baseSys on the m23 server.

\item Create an empty file (touch) with the release name of the new distribution release under /m23/data+scripts/distr/debian/debootstrap/scripts/.

\item Check, if all desired desktop environments are present in the file /m23/inc/distr/<distribution>/info.txt and add missing desktops.

\item Write a new package source list (based on a previous release) in the m23 webinterface and choose the release with the name of the previously touched file name. Hook all desktops that should be deployed with this release.

\item Make a base client install, see errors and fix them ;-)

\item Develop missing desktop installation scripts under /m23/inc/distr/<distribution>/packages/m23<desktop>Install.php.

\item Then make an installation/test/fix run with all desktop envirtonments that should be supported by the release.
\end{itemize}



\subsection{Building the packages}
\begin{itemize}
\item Generate the package template files with /mdk/m23helper/getDebianTemplates and put these under /m23/data+scripts/m23admin/packages/<distribution>/<release>/.

\item Add the sources list name in /mdk/bin/exportDBsourceslist.php.

\item Build new packages via the MDK.
\end{itemize}